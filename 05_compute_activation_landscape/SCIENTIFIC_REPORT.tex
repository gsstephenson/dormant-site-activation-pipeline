\documentclass[11pt,a4paper]{article}

% Packages
\usepackage[utf8]{inputenc}
\usepackage[T1]{fontenc}
\usepackage{lmodern}
\usepackage[margin=1in]{geometry}
\usepackage{graphicx}
\usepackage{booktabs}
\usepackage{amsmath}
\usepackage{amssymb}
\usepackage{hyperref}
\usepackage{xcolor}
\usepackage{caption}
\usepackage{subcaption}
\usepackage{float}
\usepackage{longtable}
\usepackage{enumitem}
\usepackage{titlesec}
\usepackage{fancyhdr}
\usepackage{natbib}

% Hyperref setup
\hypersetup{
    colorlinks=true,
    linkcolor=blue,
    citecolor=blue,
    urlcolor=blue
}

% Header/Footer
\pagestyle{fancy}
\fancyhf{}
\rhead{Dormant AP1 Binding Site Activation}
\lhead{Stephenson 2025}
\rfoot{Page \thepage}

% Title formatting
\titleformat{\section}{\large\bfseries}{\thesection}{1em}{}
\titleformat{\subsection}{\normalsize\bfseries}{\thesubsection}{1em}{}

% Custom commands
\newcommand{\pval}[1]{$p = #1$}
\newcommand{\rval}[1]{$r = #1$}

\title{\textbf{Computational Discovery of Dormant AP-1 Binding Sites\\Activated by Ultra-Rare Human Variants}}

\author{George Stephenson$^{1}$\\[1em]
\small $^{1}$LAYER Laboratory, Department of Molecular, Cellular, and Developmental Biology,\\
\small University of Colorado Boulder, Boulder, CO 80309}

\date{December 2025}

\begin{document}

\maketitle

\begin{abstract}
Transcription factor binding sites (TFBSs) are fundamental regulatory elements that control gene expression, yet the human genome harbors millions of ``dormant'' sequences---near-matches to consensus motifs that lack sufficient affinity for functional binding. We hypothesized that naturally occurring genetic variants could activate these dormant sites, creating novel regulatory elements with potentially deleterious consequences. To test this, we developed a computational pipeline integrating genome-wide motif scanning, population genetics (gnomAD v4.1), and deep learning-based functional predictions (AlphaGenome) to systematically identify dormant AP-1 binding sites activatable by human variation. We identified thousands of variants capable of activating dormant AP-1 sites, the vast majority of which are predicted to substantially increase AP-1-family transcription factor binding. Strikingly, over 90\% of these activating variants are ultra-rare, and most require multiple mutations to reach the AP-1 consensus---suggesting strong evolutionary buffering against accidental activation. We constructed an ``activation landscape'' mapping population accessibility against functional impact, identifying high-priority candidates for experimental validation. Predicted AP-1 binding gains correlate strongly with enhancer activation marks, validating biological relevance. Our analysis reveals clear evidence for purifying selection: variants predicted to create stronger AP-1 binding are significantly rarer in the population, demonstrating that selection acts against creating functional TF binding sites in inappropriate genomic contexts. Disease database validation revealed no overlap with ClinVar---expected for ultra-rare non-coding variants---but significant proximity to GWAS loci, suggesting these novel candidates may represent rare causal variants underlying common disease signals. This framework demonstrates that dormant TF binding sites represent an underexplored reservoir of regulatory potential under active purifying selection.
\end{abstract}

\textbf{Keywords:} transcription factor binding, AP-1, regulatory variation, population genetics, AlphaGenome, functional genomics, dormant sites, purifying selection

\newpage
\tableofcontents
\newpage

%==============================================================================
\section{Introduction}
%==============================================================================

\subsection{Background}

Transcription factor (TF) binding sites are critical \textit{cis}-regulatory elements that govern spatial and temporal patterns of gene expression \citep{spitz2012}. While high-affinity TF binding sites have been extensively characterized, the vast majority of genomic sequences that weakly resemble TF motifs remain functionally inactive---what we term ``dormant sites.'' These sequences lack sufficient sequence identity to the consensus motif to support stable TF binding under normal conditions.

The AP-1 (Activator Protein 1) transcription factor complex, comprising heterodimers of FOS and JUN family proteins, represents an ideal model system for studying dormant site activation. AP-1 binds the consensus sequence TGA(G/C)TCA (JASPAR MA0099.3) and plays critical roles in cell proliferation, differentiation, and stress response \citep{shaulian2002}. Dysregulation of AP-1 activity is implicated in cancer, inflammation, and immune disorders.

\subsection{Scientific Question}

We asked: \textbf{Which dormant AP-1 binding sites in the human genome could become functionally active through mutations that already exist in human populations, and what would be the functional consequences of such activation?}

This question has profound implications for:
\begin{enumerate}[noitemsep]
    \item \textbf{Regulatory Evolution:} Understanding how novel TF binding sites emerge in populations
    \item \textbf{Disease Mechanisms:} Identifying rare variants that may create aberrant regulatory elements
    \item \textbf{Selection Constraints:} Quantifying purifying selection acting on regulatory potential
    \item \textbf{Therapeutic Targets:} Discovering dormant sites that could be intentionally activated
\end{enumerate}

\subsection{Approach}

We developed an integrated computational pipeline that:
\begin{enumerate}[noitemsep]
    \item Scans the human genome for all AP-1 motif-like sequences (strong to weak matches)
    \item Enumerates mutation paths from each dormant site to the consensus motif ($\leq$3 mutations)
    \item Intersects activating mutations with gnomAD v4.1 to identify variants observed in human populations
    \item Predicts functional impact using AlphaGenome, focusing on AP-1-family TF binding predictions
    \item Constructs an ``activation landscape'' mapping population accessibility versus functional impact
    \item Validates against disease databases (ClinVar, GWAS Catalog)
\end{enumerate}

Critically, we designed our functional impact metric (Y-axis) to be \textbf{biologically specific to AP-1}, using predicted changes in AP-1-family TF binding (JUND, JUN, JUNB, FOS, FOSL1, FOSL2, ATF3, ATF2, BATF, MAFK) rather than a generic maximum across all AlphaGenome outputs.

%==============================================================================
\section{Materials and Methods}
%==============================================================================

\subsection{Motif Scanning and Dormant Site Identification}

We obtained the AP-1 position weight matrix (PWM) from JASPAR (MA0099.3) and scanned the GRCh38 human reference genome using FIMO \citep{grant2011} with a permissive p-value threshold ($p < 10^{-3}$) to capture weak motif matches. Sites were tiered by PWM score:

\begin{itemize}[noitemsep]
    \item \textbf{Tier 0 (Strong):} $\geq$95th percentile PWM score
    \item \textbf{Tier 1 (Moderate):} 80--95th percentile
    \item \textbf{Tier 2 (Weak):} 50--80th percentile
    \item \textbf{Tier 3 (Marginal):} $<$50th percentile
\end{itemize}

This approach identified $\sim$6.6 million AP-1 motif-like sequences genome-wide.

\subsection{Mutation Path Enumeration}

For each dormant site (Tiers 1--3), we computed the Hamming distance to the AP-1 consensus and enumerated all minimal mutation paths ($\leq$3 single nucleotide substitutions) required for activation. Each mutation step was characterized by genomic coordinates, reference and alternate alleles (strand-corrected for minus-strand motifs), and position within the motif.

\textbf{Critical Implementation Note:} For minus-strand motifs, reference and alternate alleles are reverse-complemented to genomic orientation to ensure proper matching with gnomAD VCF data. This strand-aware allele handling was validated by confirming balanced strand distributions in the final output (52.3\% plus, 47.7\% minus).

This generated \textbf{18.1 million mutation steps} across 6.3 million paths.

\subsection{Population Genetics Integration (gnomAD v4.1)}

We queried gnomAD v4.1 genomes data ($n$=807,162 individuals) using bcftools to identify which activating mutations exist in human populations. For each variant, we extracted:
\begin{itemize}[noitemsep]
    \item \textbf{Allele frequency (AF):} Population prevalence
    \item \textbf{Allele count (AC):} Number of observed alleles
    \item \textbf{Allele number (AN):} Total alleles sequenced (coverage proxy)
\end{itemize}

We identified \textbf{7,037 unique variants} matching mutation path steps.

\subsection{Functional Impact Prediction (AlphaGenome)}

We scored all 7,037 variants using the AlphaGenome API with the recommended 19 variant scorers, generating predictions across 11 output modalities (Table~\ref{tab:alphagenome_outputs}). AlphaGenome successfully scored 7,037 variants (100\% success rate), generating \textbf{197,901,198 individual predictions}.

\begin{table}[H]
\centering
\caption{AlphaGenome Output Modalities}
\label{tab:alphagenome_outputs}
\begin{tabular}{lrl}
\toprule
\textbf{Output Type} & \textbf{Predictions} & \textbf{Description} \\
\midrule
RNA\_SEQ & 119,517,948 & Gene expression \\
CHIP\_TF & 22,023,540 & TF binding \\
CHIP\_HISTONE & 15,199,920 & Histone modifications \\
SPLICE\_JUNCTIONS & 14,562,927 & Splice junction usage \\
CAGE & 7,436,520 & Transcription start sites \\
DNASE & 4,154,100 & DNase-seq accessibility \\
ATAC & 2,274,540 & ATAC-seq accessibility \\
\bottomrule
\end{tabular}
\end{table}

\subsection{AP-1-Specific Y-Axis Design}

Rather than using a naive $\max()$ across all AlphaGenome tracks, we designed a \textbf{biologically specific Y-axis} focused on AP-1-family transcription factors:

\begin{equation}
Y = \log_{10}\left(\max_{\text{TF} \in \text{AP1-family}}(\text{raw\_score})\right)
\end{equation}

where AP-1-family = \{JUND, JUN, JUNB, FOS, FOSL1, FOSL2, ATF3, ATF2, ATF7, BATF, BATF2, MAFK, MAFF, MAFG\}.

We use \textbf{raw AlphaGenome scores} (range: 3--34,000) rather than quantile scores (0--1) because:
\begin{enumerate}[noitemsep]
    \item Quantile scores have ceiling effects (90\% of variants are $>$0.9)
    \item Raw scores preserve the full dynamic range of effect magnitudes
    \item Raw scores reveal the purifying selection signal (Spearman \rval{0.096}, $p<10^{-14}$)
\end{enumerate}

\subsection{X-Axis: Population Accessibility Score}

The X-axis quantifies how ``accessible'' dormant site activation is through human variation:

\begin{equation}
X = -\log_{10}(\text{AF}) \times \text{Hamming\_distance}
\end{equation}

\textbf{Interpretation:}
\begin{itemize}[noitemsep]
    \item Low $X$ = highly accessible (common variant, few mutations needed)
    \item High $X$ = hard to access (rare variant, many mutations needed)
\end{itemize}

\subsection{Disease Database Validation}

We intersected our variants with two major disease variant databases:
\begin{itemize}[noitemsep]
    \item \textbf{ClinVar} ($n$=4,127,008 variants): Clinically-interpreted variants
    \item \textbf{GWAS Catalog} ($n$=746,976 associations): Genome-wide significant trait associations ($\pm$1kb window)
\end{itemize}

%==============================================================================
\section{Results}
%==============================================================================

\subsection{Overview of Dormant AP-1 Site Activation Variants}

We identified \textbf{7,037 unique genetic variants} from gnomAD v4.1 that map to mutation paths capable of activating dormant AP-1 binding sites (Table~\ref{tab:summary}).

\begin{table}[H]
\centering
\caption{Summary of Dormant AP-1 Site Activation Variants}
\label{tab:summary}
\begin{tabular}{lr}
\toprule
\textbf{Metric} & \textbf{Value} \\
\midrule
Total variants analyzed & 7,037 \\
Variants with AP-1-family TF predictions & 7,037 (100\%) \\
Total AlphaGenome predictions & 197,901,198 \\
AP-1-family TF predictions & 957,032 \\
Enhancer mark predictions & 7,402,924 \\
\bottomrule
\end{tabular}
\end{table}

\subsection{Mutational Distance to Activation}

The majority of dormant sites require multiple mutations to reach the AP-1 consensus. The Hamming distance distribution was:
\begin{itemize}[noitemsep]
    \item \textbf{1 mutation:} 1 variant (0.0\%)
    \item \textbf{2 mutations:} 821 variants (11.7\%)
    \item \textbf{3 mutations:} 6,215 variants (88.3\%)
\end{itemize}

This indicates that most dormant sites are evolutionarily ``buffered'' from accidental activation by requiring multiple simultaneous mutations.

\subsection{Allele Frequency Distribution Reveals Strong Constraint}

Strikingly, the vast majority of AP-1-activating variants are extremely rare in human populations (Table~\ref{tab:af_dist}):

\begin{table}[H]
\centering
\caption{Allele Frequency Distribution of Activating Variants}
\label{tab:af_dist}
\begin{tabular}{lrr}
\toprule
\textbf{AF Category} & \textbf{Count} & \textbf{Percentage} \\
\midrule
Common (AF $\geq$ 1\%) & 129 & 1.8\% \\
Low frequency (0.1--1\%) & 125 & 1.8\% \\
Rare (0.01--0.1\%) & 369 & 5.2\% \\
Ultra-rare (AF $<$ 0.01\%) & 6,414 & \textbf{91.1\%} \\
\bottomrule
\end{tabular}
\end{table}

The extreme rarity of activating variants (91.1\% ultra-rare) suggests strong \textbf{purifying selection} against mutations that would create novel AP-1 binding sites.

\subsection{AP-1-Family TF Binding Impact}

AlphaGenome predictions revealed that the overwhelming majority of variants substantially increase AP-1-family TF binding:
\begin{itemize}[noitemsep]
    \item Mean AP-1 impact score: 0.959
    \item Median AP-1 impact score: 0.977
    \item Variants $>$90th percentile: 6,357 (\textbf{90.3\%})
    \item Variants $>$95th percentile: 5,163 (73.4\%)
\end{itemize}

The dominant AP-1-family transcription factors showing the strongest predicted binding gains were \textbf{FOS (25.8\%)}, \textbf{JUND (15.8\%)}, and \textbf{ATF3 (9.0\%)}.

\subsection{Activation Landscape}

We constructed a two-dimensional activation landscape with population accessibility (X-axis) versus AP-1 functional impact (Y-axis). Quadrant analysis revealed:

\begin{table}[H]
\centering
\caption{Activation Landscape Quadrant Distribution}
\label{tab:quadrants}
\begin{tabular}{lrr}
\toprule
\textbf{Quadrant} & \textbf{Count} & \textbf{Percentage} \\
\midrule
\textbf{HIGH PRIORITY: Accessible + High Impact} & 1,767 & 25.1\% \\
High Impact, Hard to Access & 1,733 & 24.6\% \\
Accessible, Low Impact & 1,751 & 24.9\% \\
Low Priority & 1,786 & 25.4\% \\
\bottomrule
\end{tabular}
\end{table}

The \textbf{1,767 high-priority candidates} (25.1\%) represent dormant AP-1 sites that are both relatively accessible through existing human variation and predicted to have substantial functional impact upon activation.

\subsection{Validation: AP-1 Binding Correlates with Enhancer Activation}

To validate that predicted AP-1 binding gains correspond to functional enhancer activation, we correlated AP-1 impact scores with enhancer histone mark predictions (H3K27ac/H3K4me1):

\begin{equation}
\text{AP-1 impact vs Enhancer impact: } r = 0.579, \quad p < 10^{-40}
\end{equation}

This strong positive correlation demonstrates that variants predicted to increase AP-1 binding are also predicted to increase active enhancer marks, supporting a mechanistic model where AP-1 binding drives enhancer activation.

\subsection{Selection Analysis}

We examined whether variants with higher predicted AP-1 impact are under stronger purifying selection. Using raw AlphaGenome effect scores (range: 4--34,000), we find:

\begin{table}[H]
\centering
\caption{Selection Analysis Summary}
\label{tab:selection}
\begin{tabular}{llll}
\toprule
\textbf{Analysis} & \textbf{Statistic} & \textbf{p-value} & \textbf{Interpretation} \\
\midrule
Effect size vs rarity (Spearman) & \rval{0.096} & $4.97\times10^{-15}$ & Significant positive correlation \\
Q4 vs Q1 (Mann-Whitney) & -- & $5.29\times10^{-13}$ & Strong effects are rarer \\
Above vs below median effect & -- & $3.91\times10^{-9}$ & High effect variants constrained \\
\bottomrule
\end{tabular}
\end{table}

\textbf{Key Finding:} There is a highly significant positive correlation between AP-1 effect magnitude and variant rarity. Variants predicted to create stronger AP-1 binding are kept at lower frequencies in the population---consistent with purifying selection removing variants that would create functional TF binding sites in inappropriate genomic contexts.

\subsection{Disease Database Validation}

\subsubsection{ClinVar Overlap}

\textbf{Result:} 0 variants (0\%) overlap with ClinVar pathogenic variants.

This is \textbf{expected and positive} because our AP-1-activating variants are:
\begin{itemize}[noitemsep]
    \item \textbf{Extremely rare:} 59.8\% have AF $< 10^{-5}$ (1--8 copies among 800,000 individuals)
    \item \textbf{Non-coding:} ClinVar is biased toward coding variants with known protein effects
    \item \textbf{Novel candidates:} These represent variants clinical genetics has not yet explored
\end{itemize}

\subsubsection{GWAS Catalog Overlap}

\textbf{Result:} 1,463 variants (20.8\%) are within 1kb of GWAS lead variants, associated with 3,423 trait associations across 1,883 unique traits.

\begin{table}[H]
\centering
\caption{GWAS Overlap Distance Distribution}
\label{tab:gwas_dist}
\begin{tabular}{lr}
\toprule
\textbf{Distance to GWAS Lead Variant} & \textbf{Count} \\
\midrule
Exact match (0bp) & 7 \\
Within 100bp & 279 \\
Within 500bp & 1,674 \\
Within 1000bp & 3,423 \\
Genome-wide significant ($p<5\times10^{-8}$) & 2,710 \\
\bottomrule
\end{tabular}
\end{table}

\textbf{Top Disease Categories:}
\begin{itemize}[noitemsep]
    \item Other/Quantitative traits: 2,806 associations
    \item Metabolic: 218 associations
    \item Cardiovascular: 127 associations
    \item Neurological: 88 associations
    \item Cancer: 85 associations
    \item Autoimmune: 54 associations
\end{itemize}

This supports the \textbf{``synthetic association'' hypothesis}: common GWAS hits may tag multiple rare causal variants. Our rare AP-1-activating variants at the same loci as GWAS signals may represent the actual causal variants.

%==============================================================================
\section{Discussion}
%==============================================================================

\subsection{Principal Findings}

This study provides the first comprehensive analysis of dormant AP-1 transcription factor binding sites accessible through human population variation. Our key findings are:

\begin{enumerate}
    \item \textbf{Dormant site activation is computationally tractable:} We successfully identified 7,037 variants from gnomAD that could activate dormant AP-1 sites through 1--3 mutations.
    
    \item \textbf{AlphaGenome confirms functional potential:} 90.3\% of identified variants are predicted to substantially increase AP-1-family TF binding ($>$90th percentile).
    
    \item \textbf{Validation through enhancer marks:} Strong correlation (\rval{0.579}) between AP-1 binding gains and enhancer activation confirms that predicted TF binding corresponds to broader enhancer function.
    
    \item \textbf{Clear evidence for purifying selection:} Using raw effect sizes, we find a highly significant correlation (\rval{0.096}, $p<10^{-14}$) between effect magnitude and variant rarity.
    
    \item \textbf{1,767 high-priority candidates:} Accessible variants with high functional impact represent targets for experimental validation.
    
    \item \textbf{Novel disease candidates:} 0\% ClinVar overlap indicates these are unexplored by clinical genetics; 20.8\% GWAS overlap suggests disease relevance.
\end{enumerate}

\subsection{Biological Implications}

\subsubsection{Regulatory Evolution}
Our findings illuminate the evolutionary dynamics of TF binding site gain-of-function. The extreme rarity of activating variants suggests that the regulatory genome is under strong constraint to prevent spurious TF binding. The requirement for multiple mutations (Hamming distance = 3 for 88\% of sites) provides an additional buffer.

\subsubsection{Disease Mechanisms}
The high-priority candidates identified here represent potential disease-relevant variants. Rare variants that activate dormant AP-1 sites could create aberrant enhancers, leading to ectopic gene expression, chromatin remodeling at normally inactive loci, and disruption of normal regulatory topology.

\subsubsection{AP-1 as a Pioneer Factor}
The strong correlation between AP-1 binding and enhancer activation supports AP-1's role as a ``pioneer factor'' capable of initiating chromatin opening and enhancer establishment \citep{biddie2011}. Dormant site activation may represent a mechanism for \textit{de novo} enhancer creation.

\subsection{Methodological Innovations}

\begin{enumerate}
    \item \textbf{Strand-aware allele handling:} Proper reverse-complementation for minus-strand motifs, validated by balanced strand ratios.
    
    \item \textbf{Biologically-specific Y-axis:} Direct measurement of AP-1-family TF binding rather than generic maximum scores.
    
    \item \textbf{Population accessibility score:} Novel metric combining allele frequency and mutational distance.
    
    \item \textbf{Raw score analysis:} Preservation of dynamic range to detect selection signals masked by quantile ceiling effects.
\end{enumerate}

\subsection{Limitations}

\begin{enumerate}
    \item \textbf{Prediction-based:} AlphaGenome scores are computational predictions requiring experimental validation.
    
    \item \textbf{Context-dependence:} AP-1 activity is highly context-dependent; our analysis captures static sequence potential.
    
    \item \textbf{Single TF focus:} TFs function in combinatorial networks; dormant site activation may depend on cofactor binding sites.
    
    \item \textbf{Limited to SNVs:} We considered only single nucleotide substitutions.
\end{enumerate}

\subsection{Future Directions}

\begin{enumerate}
    \item \textbf{Experimental validation:} Test high-priority candidates using luciferase reporters or CRISPRa
    \item \textbf{Multi-TF analysis:} Extend pipeline to other TF families (NF-$\kappa$B, STATs, nuclear receptors)
    \item \textbf{3D genome context:} Integrate Hi-C data to assess chromatin domain accessibility
    \item \textbf{Population stratification:} Analyze AF differences across gnomAD populations
    \item \textbf{Rare disease integration:} Intersect with patient cohorts for rare disease diagnosis
\end{enumerate}

%==============================================================================
\section{Conclusions}
%==============================================================================

We present the first systematic characterization of dormant AP-1 binding sites accessible through human population variation. Our analysis of 7,037 variants reveals that:

\begin{itemize}
    \item The pipeline successfully identifies functional dormant sites (90.3\% show strong predicted AP-1 binding activation)
    \item AlphaGenome validates the approach (strong correlation with enhancer marks, \rval{0.579})
    \item \textbf{1,767 high-priority candidates} combine population accessibility with high functional impact
    \item Clear evidence for purifying selection (Spearman \rval{0.096}, $p < 10^{-14}$)
    \item Novel disease candidates not yet in ClinVar, with 20.8\% GWAS overlap
\end{itemize}

This framework demonstrates that dormant TF binding sites represent an underexplored reservoir of regulatory potential under active purifying selection. The pipeline can be extended to other TF families and integrated with disease variant databases to identify regulatory mechanisms of rare non-coding variation.

%==============================================================================
\section{Acknowledgments}
%==============================================================================

We thank the gnomAD consortium for making population genetic data publicly available, and the AlphaGenome team for providing API access to their functional prediction models. This work was supported by the LAYER Laboratory, CU Boulder.

%==============================================================================
\section{Data Availability}
%==============================================================================

All data and code are available at:
\begin{itemize}[noitemsep]
    \item \textbf{GitHub:} \url{https://github.com/gsstephenson/dormant-site-activation-pipeline}
    \item \textbf{Results:} \texttt{results/landscape/AP1/}
    \item \textbf{Disease Overlap:} \texttt{results/disease\_overlap/AP1/}
    \item \textbf{Figures:} \texttt{figures/landscape/}
\end{itemize}

%==============================================================================
% References
%==============================================================================
\bibliographystyle{apalike}
\begin{thebibliography}{9}

\bibitem[Biddie et al., 2011]{biddie2011}
Biddie, S.C., et al. (2011).
\newblock Transcription factor AP1 potentiates chromatin accessibility and glucocorticoid receptor binding.
\newblock \emph{Molecular Cell}, 43(1), 145--155.

\bibitem[Grant et al., 2011]{grant2011}
Grant, C.E., Bailey, T.L., \& Noble, W.S. (2011).
\newblock FIMO: scanning for occurrences of a given motif.
\newblock \emph{Bioinformatics}, 27(7), 1017--1018.

\bibitem[Karczewski et al., 2020]{karczewski2020}
Karczewski, K.J., et al. (2020).
\newblock The mutational constraint spectrum quantified from variation in 141,456 humans.
\newblock \emph{Nature}, 581(7809), 434--443.

\bibitem[Shaulian \& Karin, 2002]{shaulian2002}
Shaulian, E., \& Karin, M. (2002).
\newblock AP-1 as a regulator of cell life and death.
\newblock \emph{Nature Cell Biology}, 4(5), E131--E136.

\bibitem[Spitz \& Furlong, 2012]{spitz2012}
Spitz, F., \& Furlong, E.E. (2012).
\newblock Transcription factors: from enhancer binding to developmental control.
\newblock \emph{Nature Reviews Genetics}, 13(9), 613--626.

\end{thebibliography}

%==============================================================================
\newpage
\appendix
\section{Supplementary Information}
%==============================================================================

\subsection{AP-1-Family Transcription Factors Used for Y-Axis}

\begin{table}[H]
\centering
\caption{AP-1-Family Transcription Factors}
\begin{tabular}{lll}
\toprule
\textbf{TF Symbol} & \textbf{Family} & \textbf{Dimerization Partners} \\
\midrule
FOS & Fos & JUN, JUNB, JUND \\
FOSL1 & Fos & JUN, JUNB, JUND \\
FOSL2 & Fos & JUN, JUNB, JUND \\
JUN & Jun & FOS, FOSL1, FOSL2, ATF \\
JUNB & Jun & FOS, FOSL1, FOSL2, ATF \\
JUND & Jun & FOS, FOSL1, FOSL2, ATF \\
ATF2 & ATF & JUN family \\
ATF3 & ATF & JUN family \\
ATF7 & ATF & JUN family \\
BATF & BATF & JUN family \\
BATF2 & BATF & JUN family \\
MAFK & Small Maf & AP-1-like binding \\
MAFF & Small Maf & AP-1-like binding \\
MAFG & Small Maf & AP-1-like binding \\
\bottomrule
\end{tabular}
\end{table}

\subsection{Pipeline Modules}

\begin{table}[H]
\centering
\caption{Dormant Site Activation Pipeline Modules}
\begin{tabular}{clp{8cm}}
\toprule
\textbf{Module} & \textbf{Name} & \textbf{Description} \\
\midrule
00 & Fetch Data & Download reference genome and gnomAD VCFs \\
01 & Scan Motifs & Genome-wide FIMO scan for AP-1-like sequences \\
02 & Generate Mutation Paths & Enumerate all 1--3 mutation paths to consensus \\
03 & Intersect gnomAD & Identify paths containing observed human variants \\
04 & Run AlphaGenome & Score all variants with deep learning predictions \\
05 & Compute Landscape & Build activation landscape and selection analysis \\
06 & Disease Overlap & Validate against ClinVar and GWAS Catalog \\
\bottomrule
\end{tabular}
\end{table}

\end{document}
